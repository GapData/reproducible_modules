\begin{table}[!h]

\caption{\label{tab:}Some examples of tracks of training modules for sample trainees}
\centering
\fontsize{10}{12}\selectfont
\begin{tabular}[t]{>{\centering\arraybackslash}p{20em}ccc}
\toprule
\rotatebox{45}{} & \rotatebox{45}{\makecell[l]{Long description of\\the student}} & \rotatebox{45}{Module 2} & \rotatebox{45}{\makecell[l]{Another very long\\description of the\\research associate}}\\
\midrule
\addlinespace[0.3em]
\multicolumn{4}{l}{\textbf{Very long sequence name for sequence 1}}\\
\hspace{1em}\tabitem Separating data recording and analysis & \cellcolor{pink}{Yes} & \cellcolor{white}{No} & \cellcolor{white}{No}\\
\hspace{1em}\tabitem Principals and power of structured data formats & \cellcolor{pink}{Yes} & \cellcolor{pink}{Yes} & \cellcolor{white}{No}\\
\hspace{1em}\tabitem The 'tidy' data format: an implementation of a structured data format & \cellcolor{pink}{Yes} & \cellcolor{white}{No} & \cellcolor{pink}{Yes}\\
\hspace{1em}\tabitem Designing templates for tidy data collection & \cellcolor{pink}{Yes} & \cellcolor{white}{No} & \cellcolor{white}{No}\\
\hspace{1em}\tabitem Example: Creating a template for data collection & \cellcolor{pink}{Yes} & \cellcolor{white}{No} & \cellcolor{white}{No}\\
\hspace{1em}\tabitem Power of using a single 'Project' directory for storing and tracking research project files & \cellcolor{pink}{Yes} & \cellcolor{white}{No} & \cellcolor{pink}{Yes}\\
\hspace{1em}\tabitem Creating 'Project' templates & \cellcolor{pink}{Yes} & \cellcolor{white}{No} & \cellcolor{white}{No}\\
\hspace{1em}\tabitem Example: Creating a 'Project' template & \cellcolor{pink}{Yes} & \cellcolor{pink}{Yes} & \cellcolor{white}{No}\\
\hspace{1em}\tabitem Harnessing version control for transparent data recording & \cellcolor{pink}{Yes} & \cellcolor{white}{No} & \cellcolor{pink}{Yes}\\
\hspace{1em}\tabitem Using git and GitLab to enhance the reproducibility of collaborative research & \cellcolor{pink}{Yes} & \cellcolor{white}{No} & \cellcolor{white}{No}\\
\addlinespace[0.3em]
\multicolumn{4}{l}{\textbf{Sequence 2}}\\
\hspace{1em}\tabitem Principals and benefits of scripted pre-processing of experimental data & \cellcolor{pink}{Yes} & \cellcolor{white}{No} & \cellcolor{white}{No}\\
\hspace{1em}\tabitem Introduction to R code scripts & \cellcolor{pink}{Yes} & \cellcolor{pink}{Yes} & \cellcolor{white}{No}\\
\hspace{1em}\tabitem Simplify scripted pre-processing through R's 'tidyverse' tools & \cellcolor{pink}{Yes} & \cellcolor{white}{No} & \cellcolor{pink}{Yes}\\
\hspace{1em}\tabitem Complex data types in experimental data pre-processing & \cellcolor{pink}{Yes} & \cellcolor{white}{No} & \cellcolor{white}{No}\\
\hspace{1em}\tabitem Complex data types in R and Bioconductor & \cellcolor{pink}{Yes} & \cellcolor{white}{No} & \cellcolor{white}{No}\\
\hspace{1em}\tabitem Example: Converting from complex data types to 'tidy' data formats & \cellcolor{pink}{Yes} & \cellcolor{white}{No} & \cellcolor{white}{No}\\
\hspace{1em}\tabitem Introduction to reproducible data pre-processing protocols & \cellcolor{pink}{Yes} & \cellcolor{pink}{Yes} & \cellcolor{white}{No}\\
\hspace{1em}\tabitem Introduction to RMarkdown as a tool for creating reproducible data pre-processing protocols & \cellcolor{pink}{Yes} & \cellcolor{white}{No} & \cellcolor{pink}{Yes}\\
\hspace{1em}\tabitem Example: Creating a reproducible data pre-processing protocol & \cellcolor{pink}{Yes} & \cellcolor{white}{No} & \cellcolor{white}{No}\\
\bottomrule
\end{tabular}
\end{table}
