
\begin{landscape}\begingroup\fontsize{10}{12}\selectfont
\rowcolors{2}{white}{gray!6}

\begin{longtable}[t]{>{\bfseries\raggedright\arraybackslash}p{10em}>{\raggedright\arraybackslash}p{28em}>{\raggedright\arraybackslash}p{14em}>{\raggedright\arraybackslash}p{3em}>{\raggedright\arraybackslash}p{14em}}
\caption{\label{tab:}\label{tab:content_one} Modules for the first sequence, \textbf{'Improving the Reproducibility of Experimental Data Recording'}. The color of each module's title indicates whether the module focuses on \textbf{Principles} (blue), \textbf{Implementation} (red), or \textbf{Case study examples} (black). This table is continued over several pages.}\\
\hiderowcolors
\toprule
Module title & Description of module content & Objectives (After taking the module, the trainee can ...) & Video length & Extra educational materials\\
\midrule
\endfirsthead
\caption[]{\label{tab:content_one} Modules for the first sequence, \textbf{'Improving the Reproducibility of Experimental Data Recording'}. The color of each module's title indicates whether the module focuses on \textbf{Principles} (blue), \textbf{Implementation} (red), or \textbf{Case study examples} (black). This table is continued over several pages. \textit{(continued)}}\\
\toprule
Module title & Description of module content & Objectives (After taking the module, the trainee can ...) & Video length & Extra educational materials\\
\midrule
\endhead
\
\endfoot
\bottomrule
\endlastfoot
\showrowcolors
\textcolor{blue}{\textbf{Separating data recording and analysis}} & Many biomedical laboratories currently use spreadsheets, with embedded macros, 
      to both record and analyze experimental data. This practice impedes the transparency
      and reproducibility of both data recording and data analysis. In this module, we 
      will describe this common practice and explain how it impedes the transparency and
      reproducibility of data recording and analysis. We will then outline alternative
      approaches that separate the steps of data recording and data analysis and explain
      how these alternative approaches can improve the reproducibility of biomedical 
      research. & \tabitem Explain the difference between data recording and data analysis 

     \tabitem Understand why collecting data on spreadsheets with embedded macros
        impedes transparency and reproducibility 

      \tabitem List alternative approaches that separate data recording and data analysis to 
        improve transparency and reproducibility & 15 & \tabitem Discussion questions about data recording approaches the trainee has 
      previously used in research projects and the benefits
      and limitations of those approaches in terms of data transparency and 
      reproducibility 

    \tabitem Short audio recording of two Co-Is giving their
      own answers to these discussion questions\\
\textcolor{blue}{\textbf{Principles and power of structured data formats}} & The format in which experimental data is recorded can have a large influence
      on how easy and likely it is to implement reproducibility tools in later stages of
      data pre-processing, analysis, and visualization. Recording data in a 'structured'
      format brings many benefits for later stages of the research process, 
      especially in terms of improving reproducibility.
      In this module, we will explain what makes a dataset 'structured' and
      why this format is a powerful tool for reproducible research. & \tabitem List the characteristics of a structured data format 

      \tabitem Describe how using a structured data format when recording experimental 
      data can improve the transparency and reproducibility of research

      \tabitem Outline other benefits of using a structured format when recording data & 10 & \tabitem Applied exercise: For example datasets, specify whether each is in a 
      structured data format and, in cases where it is not, draft a structured
      format that could be used to record the data 

    \tabitem Video walking trainees 
      through solutions to the applied exercise\\
\textcolor{red}{\textbf{The 'tidy' data format: an implementation of a structured data format}} & The 'tidy' data format is one implementation of a structured data format that
  was introduced in a 2014 paper and has since quickly 
  gained popularity among statisticians and data scientists. By consistently 
  using this data format, researchers have found they can employ combinations 
  of simple, generalizable tools to perform complex tasks in data processing, 
  analysis, and visualization. In this module, we will explain what characteristics determine
  if a dataset is 'tidy' and how use of the 'tidy' implementation of a structure 
  data format can improve the ease and efficiency
  of 'Team Science', including collaborations with statisticians. & \tabitem List characteristics defining the  
    the 'tidy' structured data format 

  \tabitem Explain the difference between the ideas of a structured data format (general 
    concept) and the 'tidy' data format (one implementation of that general format
    that is now particularly popular in data analysis) & 15 & \tabitem Quiz questions: For example datasets, correctly identify which of the 'tidy'
  data principles the dataset has or lacks 

  \tabitem Video giving answers and explanations
  for quiz questions, including showing 'tidy' versions of each example dataset 

  \tabitem Link to paper that established the 'tidy' data format\\
\textcolor{red}{\textbf{Designing templates for tidy data collection}} & This module will move from the principles of the 'tidy' data format to the 
      practical details of designing a 'tidy' data format to use when collecting 
      experimental data. We will describe common issues that prevent many real datasets from
      experimental research projects from being 'tidy' and show how these issues
      can be avoided when deciding the format in which to record experimental data.
      We will also provide rubrics and a checklist to help determine if a 
      data collection template complies with a 'tidy' format. & \tabitem Identify characteristics that keep a dataset from being 'tidy'
      
      \tabitem Convert data from an 'untidy' to a 'tidy' format & 20 & \tabitem Applied exercise: For an 'untidy' dataset, identify what 
      characteristics keep it from being 'tidy', and convert design a 'tidy' format

  \tabitem Video providing a detailed solution to the applied exercise\\
\textcolor{black}{\textbf{Example: Creating a template for 'tidy' data collection}} & In this module, we will walk through an example of creating a template to collect
      data in a 'tidy' format for a laboratory-based research project. As an example,
      we will use a research project headed by one of our Co-Is on drug efficacy in 
      murine tuberculosis models. We will walk through the 'untidy' format 
      initially used to collect data for this project, explain how this format 
      differed from a 'tidy' format, and show how we changed the format to be 'tidy'.
      Finally, we will show examples of how the experimental data can easily be 
      cleaned, analyzed, and visualized using reproducible tools once it is in a 
      'tidy' format. & \tabitem Understand how the principles of 'tidy' data can be applied 
      when recording experimental
      data for a real, complex research project;

      \tabitem List some advantages of using a 'tidy' data format for the example project & 15 & \tabitem Discussion questions, including listing examples of how experimental datasets
      the trainee has previously worked with or is currently working with are 'untidy' and
      how they could be converted to a 'tidy' format 

    \tabitem Short audio recording of two Co-Is giving their
      own answers to these discussion questions\\
\addlinespace
\textcolor{blue}{\textbf{Power of using a single structured 'Project' directory for storing and tracking research project files}} & To improve the computational reproducibility of a research project, researchers
      can use a single 'Project' directory to collectively store 
      all research data (raw and pre-processed), meta-data, code for data pre-processing,
      and research products further along the research pipeline (e.g., paper drafts, 
      figures, code for data analysis). In this 
      module, we will explain how using this practice from the 
      start of a research project improves the reproducibility of the projects, as well
      as facilitates other tools to improve reproducibility,
      including version control. Finally, we will 
      list some of the common components and subdirectories to include
      in the structure of a 'Project' directory, including subdirectories for raw and
      pre-processed experimental data. & \tabitem Describe a 'Project' directory, including common components and subdirectories 

      \tabitem List how collecting all research data and other files related 
      to a research project in a single 'Project' directory
      improves the reproducibility of a research project 

      \tabitem Describe how experimental data collection can be integrated with a
      research 'Project' directory & 20 & \tabitem Quiz questions: Test the trainee's understanding of a structured
      'Project' directory, what common components it may include, and the benefits
      of structuring research project files 
      within a single 'Project' directory from the beginning of the
      research project 

      \tabitem Video with detailed answers and discussion of quiz questions\\
\textcolor{red}{\textbf{Creating 'Project' templates}} & Researchers can use RStudio's 'Projects' interface to implement the structured
      collection of files for a research project in a single directory, with the added
      benefits that this interface facilitates use of version control. 
      Researchers can gain even more benefits, in terms of improving both the reproducibility
      and efficiency of research, by using a consistent structure for the 'Project' 
      directories for all of the research projects for a research group. We will demonstrate 
      how to implement structured project directories through RStudio,
      as well as how RStudio enables the creation of a template for all of a 
      research group's 'Project' directories, so a new project can be initialized
      with a skeleton directory that follows a directory format established
      by the research group. & \tabitem Be able to create a structured `Project` directory within RStudio 
      to use to consistently and reproducibly manage all files for a research project

     \tabitem Understand how RStudio can be used to create a template
      to use to create consistently-structured research 'Project' directories & 25 & \tabitem Discussion questions, including descriptions of how the trainee has saved and
      tracked research project files for previous research projects and what barriers,
      if any, these practices introduced in terms of the reproducibility and efficiency
      of research 

    \tabitem Short audio recording of two Co-Is discussing their answers to these questions\\
\textcolor{black}{\textbf{Example: Creating a 'Project' template}} & In this module, we will walk through a real example, based on the experiences of
      one of our Co-Is, of establishing the format 
      for a research group's 'Project' template, creating that template using RStudio,
      and initializing a new research project directory using the created template.
      This example will be from a laboratory-based research group that studies the efficacy of 
      tuberculosis drugs in a murine model. & \tabitem Create a 'Project' template in RStudio to use to initialize 
      consistently-formatted 'Project' directories to store all files related to 
      a research project
  
      \tabitem Initialize a new 'Project' directory using this template & 15 & \tabitem Applied exercise: Create and save a 'Project' 
      template that meets specifications provided for an example research group; 

     \tabitem Video demonstrating a detailed solution 
      to the applied exercise.\\
\textcolor{blue}{\textbf{Harnessing version control for transparent data recording}} & As a research project progresses, a typical practice in many experimental 
      research groups is to save new versions of files (e.g., 'draft1.doc', 'draft2.doc'),
      so that changes can be reverted. However, this practice 
      leads to an explosion of files, and it becomes hard to track 
      which files represent the 'current' state of a project. Version control allows
      researchers to edit and change research project files more cleanly, while maintaining
      the power to 'backtrack' to previous versions. Further, with version control,
      messages can be included to explain any changes.
      In this module, we will explain what version
      control is and how it can be used in research projects to improve the transparency 
      and reproducibility of research, particularly for transparent data recording. & \tabitem Describe version control and what it does 

      \tabitem Explain how version control can be used to improve reproducibility at 
      the data recording stage of research & 10 & \tabitem Discussion questions, including discussion of how the trainee has 
      managed evolving research project files in previous projects and any barriers
      those practices introduced in conducting efficient and reproducible research 

      \tabitem Short audio recording of two Co-Is giving their
      own answers to these discussion questions\\
\textcolor{blue}{\textbf{Enhance the reproducibility of collaborative research with version control platforms}} & Once a researcher has learned to use git on their own 
      computer for local version control, they can begin using version control 
      platforms (e.g., GitLab, GitHub) to collaborate with others in their research
      group while keeping the project under version control. These platforms allow
      the all collaborators to share a current version of a project directory 
      (similar to Dropbox), but in a way that allows easy use of version control 
      and that is more efficient for exploring (and, when necessary, undoing) the changes 
      each team member has made to project files. In this module, we will describe 
      why a research team may want to use a version control platform like GitLab 
      to work collaboratively on a project. & \tabitem List the benefits of using a version control platform like GitLab, rather 
      than Dropbox, to share project files, 
      particularly in terms of improving transparency and reproducibility 

     \tabitem Describe the difference between version control (e.g., git) and 
      a version control platform (e.g., GitLab) & 10 & \tabitem Discussion questions: Describe how you have shared research project 
    files in past research projects---email? Dropbox? Department servers?

    \tabitem Short audio file with two Co-Is discussing their answers\\
\textcolor{red}{\textbf{Using git and GitLab to implement version control}} & For many years, use of version control required use of the command line,
  limiting its accessibility to researchers with limited programming experience.
  However, graphical interfaces have removed this barrier, and RStudio has 
  particularly user-friendly tools for implementing version control.
  In this module, we will show how to use 
  \textit{git} through RStudio's user-friendly interface and how to connect from a local
  computer to \textit{GitLab} through RStudio. & \tabitem Understand how to set up and use \textit{git} through RStudio's interface 

  \tabitem Understand how to connect with \textit{GitLab} through RStudio to collaborate on  
  research projects while maintaining version control & 20 & \tabitem Applied exercise: Use RStudio to 
  initialize \textit{git} version control for a directory 
  and to make several tracked changes. Create a matching \textit{GitLab} repository and use
  RStudio to push local changes to this GitLab version of the directory

  \tabitem Video 
  walking trainees through a detailed solution to the exercise\\*
\end{longtable}
\rowcolors{2}{white}{white}\endgroup{}
\end{landscape}
