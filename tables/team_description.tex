\rowcolors{2}{gray!6}{white}
\begin{table}[!h]

\caption{\label{tab:}\label{tab:team_description} Principal and co-investigators on our project team.}
\centering
\resizebox{\linewidth}{!}{
\fontsize{10}{12}\selectfont
\begin{tabular}[t]{>{\raggedright\arraybackslash}p{14em}>{\raggedright\arraybackslash}p{35em}}
\hiderowcolors
\toprule
Person / role & Description\\
\midrule
\showrowcolors
\textbf{Brooke Anderson}

  Principal Investigator
  
  \textit{Assistant Professor,}
  
  \textit{Dept of Environmental \& Radiological Health Sciences} & Dr. Anderson is an expert in R 
  programming and has created and published several open-source R packages, in 
  particular to facilitate environmental epidemiological research. She has experience 
  creating R programs to work with large data, including climate model output and 
  large weather datasets, as well as programs that interface with open web-based datasets.
  She is the co-instructor of a series of Massive Open Online Courses on 
  \textit{Mastering Software Development in R} through Coursera and an associated open
  online book.\\
\textbf{Michael Lyons}

  Co-Investigator
  
  \textit{Assistant Professor,}
  
  \textit{Dept. of Microbiology, Immunology \& Pathology} & Dr. Lyons works on the computational biology and
  pharmacology of tuberculosis infection and treatment in experimental animal
  models and tuberculosis patients. Prior to joining CSU full-time in 2011, he was a
  software engineer in the computer industry for 12 years, and prior to that, a
  theoretical physicist. Through a K25 award, he obtained significant classroom
  and hands-on training and exposure to laboratory methods related to drug and
  vaccine development for tuberculosis, providing him with a solid understanding of how
  preclinical and clinical data are used for evidence-based decision making in the
  biomedical sciences. He is highly attuned to the problems that this project aims
  to address, and he has a clear understanding of the practical limitations and
  challenges for both the laboratory scientist and data analyst. He uses R daily in his academic research\\
\textbf{Mercedes Gonzalez-Juarrero} 
  
  Co-Investigator
  
  \textit{Associate Professor,}
  
  \textit{Dept. of Microbiology, Immunology \& Pathology} & Dr. Gonzalez-Juarrero studies the basic nature of the
  cell mediated immune response to mycobacteria infections. During the last ten
  years, her research group has undertaken studies to investigate the emergence of
  immunosuppression during pulmonary tuberculosis, with the primary goal of
  learning how and where to target the latently infected host to fully recover the
  antimicrobial activity of the infected cell, and how to use this information in
  the context of current chemotherapeutic and multidrug resistant tuberculosis infections.
  Dr. Gonzalez-Juarrero became particularly interested in how to improve the
  reproducibility, transparency, and efficiency of experimental data recording
  within her research projects when she attended Dr. Anderson's CSU course on
  \textit{R Programming for Research} in Fall 2017 and learned about the
  principles of structured data formats, including the ``tidy" data format now
  popular with statisticians, and she has begun implementing these principles in her research laboratory.\\
\textbf{Marcela Henao-Tamayo}
  
  Co-Investigator
  
  \textit{Assistant Professor,}
  
  \textit{Dept. of Microbiology, Immunology \& Pathology,}
  
  \textit{Co-Director of CSU-Flow Cytometry Facility} & Dr. Henao-Tamayo studies the immunopathogenesis of
  tuberculosis using animal models to evaluate the role of different types of T
  cells and myeloid-derived cells in tuberculosis and Bacille
  Calmette Guerin vaccination. She has tested numerous
  vaccine candidates evaluating the immune response they elicit in association
  with protection against tuberculosis disease. She is interested in how existing
  tools for computational reproducibility can be applied to data recording and
  pre-processing in her own research laboratory, and she and Dr. Anderson (PI)
  co-advise a graduate student who is integrating open-source R software into the
  regular practice of Dr. Henao-Tamayo's research work, including through
  implementation of reproducible automated gating of flow cytometry data.\\
\textbf{Gregory Robertson}
  
  Co-Investigator
  
  \textit{Assistant Professor,}
  
  \textit{Dept. of Microbiology, Immunology \& Pathology} & Dr. Robertson has more than 20 years of classical and clinical microbiology
  experience, with an emphasis in antibacterial discovery and mode-of-action studies
  for novel and existing classes of antimicrobials. This includes efforts in
  academia, and also with larger pharmaceutical corporations (Eli Lilly and Co)
  and smaller bio-pharmaceutical groups (Cumbre Pharmaceuticals). His current
  research is focused on \textit{Mycobacterium tuberculosis} host-pathogen
  interactions and the development and application of novel preclinical animal
  models to further anti-tuberculosis drug development and evaluate drug
  resistance. In the context of improving reproducibility in biomedical research,
  Dr. Robertson is particularly passionate about the perils of using spreadsheets
  with embedded formulas as a tool for recording and analyzing experimental data.\\
\bottomrule
\end{tabular}}
\end{table}
\rowcolors{2}{white}{white}
