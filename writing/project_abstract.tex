\documentclass[pdftex,english,11pt,parskip=half]{scrartcl}
\usepackage{palatino}
\usepackage{mathpazo}
\usepackage[margin=0.7in]{geometry}
%\usepackage{parskip}
\usepackage[compact]{titlesec}
\usepackage{amsmath,amssymb}
\usepackage{graphicx}
\usepackage{babel}
\usepackage{framed}
\usepackage{wrapfig}
\usepackage{subfig}
\usepackage[labelfont=bf,font=small,format=plain]{caption}
\usepackage{doi}
\usepackage{booktabs}
\usepackage{longtable}
\usepackage{multirow}
\usepackage[table]{xcolor}
\usepackage{wrapfig}
\usepackage{colortbl}
\usepackage{pdflscape}
\usepackage{tabu}
\usepackage{threeparttable}
\usepackage{threeparttablex}
\usepackage{array}
\usepackage[normalem]{ulem}
\usepackage{makecell}
\usepackage{float}
%\usepackage[authoryear]{natbib}
\usepackage[numbers]{natbib}
\usepackage{url,hyperref,color}
\definecolor{darkblue}{rgb}{0.0,0.0,0.75}
\hypersetup{colorlinks,breaklinks,
            linkcolor=darkblue,urlcolor=darkblue,
            anchorcolor=darkblue,citecolor=darkblue}
\newcommand{\fixme}[1]{{\color{red} #1}}
\renewcommand\thesection{\Alph{section}}
\renewcommand{\familydefault}{\sfdefault}
\newcommand{\tabitem}{~~\llap{\textbullet}~~}
\begin{document}
\addtokomafont{section}{\large}
\def\bf{\normalfont\bfseries}
\pagestyle{empty}

{\large \textbf{Project Abstract:}}

We propose to develop training modules for improving the reproducibility of experimental data recording and pre-processing in scientific research. We aim to ensure these training modules are useful to laboratory-based researchers, who may have less prior training in open source software tools for reproducible research than researchers from biostatistics, epidemiology, and other disciplines. To ensure this, we will feature in these training modules examples from microbiology and immunology. We will create two sequences of modules that focus on improving computational reproducibility in the recording and pre-processing of experimental data, each containing approximately 12 modules, each featuring 5--30 minute videos, with supplemental online text, references, and practice exercises. The first sequence will be ``Improving the Reproducibility of Experimental Data Recording", and it will include modules on ... . The second sequence will be ``Improving the Reproducibility of Experimental Data Pre-Processing", and it will include modules on ... . These two sequences of training modules will be collectively published as an open online book using the \textit{bookdown} technology. Each module will form a chapter of this book, and will feature an embedded YouTube video of 5--30 minutes, with accompanying text in the book to provide trainees with a more detailed written reference they can refer to after completing the video module. Each module's chapter will conclude with practical exercises or open discussion questions to complement the material taught in the video. To ensure this material is completely free and open to researchers in the United States, we will publish this online book and the videos under a Creative Commons license.

\end{document}