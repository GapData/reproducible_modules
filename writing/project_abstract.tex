\documentclass[pdftex,english,11.5pt,parskip=half]{scrartcl}
\usepackage{palatino}
\usepackage{mathpazo}
\usepackage[margin=0.55in]{geometry}
%\usepackage{parskip}
\usepackage[compact]{titlesec}
\usepackage{amsmath,amssymb}
\usepackage{graphicx}
\usepackage{babel}
\usepackage{framed}
\usepackage{wrapfig}
\usepackage{subfig}
\usepackage[labelfont=bf,font=small,format=plain]{caption}
\usepackage{doi}
\usepackage{booktabs}
\usepackage{longtable}
\usepackage{multirow}
\usepackage[table]{xcolor}
\usepackage{wrapfig}
\usepackage{colortbl}
\usepackage{pdflscape}
\usepackage{tabu}
\usepackage{threeparttable}
\usepackage{threeparttablex}
\usepackage{array}
\usepackage[normalem]{ulem}
\usepackage{makecell}
\usepackage{float}
%\usepackage[authoryear]{natbib}
\usepackage[numbers]{natbib}
\usepackage{url,hyperref,color}
\definecolor{darkblue}{rgb}{0.0,0.0,0.75}
\hypersetup{colorlinks,breaklinks,
            linkcolor=darkblue,urlcolor=darkblue,
            anchorcolor=darkblue,citecolor=darkblue}
\newcommand{\fixme}[1]{{\color{red} #1}}
\renewcommand\thesection{\Alph{section}}
\renewcommand{\familydefault}{\sfdefault}
\newcommand{\tabitem}{~~\llap{\textbullet}~~}
\begin{document}
\addtokomafont{section}{\large}
\def\bf{\normalfont\bfseries}
\pagestyle{empty}

{\large \textbf{Project Abstract:}}

% 30 lines of text

An essential aspect of both translational medicine and human 
health-based team science is the integration of biomedical research 
results that are generated by multiple laboratories using widely varying 
experimental systems and data analysis methods. Beyond the uncertainties of 
experimental design and measurement, there are several critical points in 
the subsequent research workflow where lack of rigor and transparency 
may compromise the reproducibility of these laboratory results.  In this proposal, 
we focus on data recording and data pre-processing as key steps 
of research workflows, and we propose to develop training modules that 
provide general principles, software tools, and exercises to broadly 
enhance data reproducibility of these steps in biomedical research. We aim to
ensure these training modules are clear, relevant, and useful to laboratory-based 
researchers, whose attention is rather to their experimental technique and collection 
of accurate data, and who may have little or no background in the use of general 
purpose software tools.  To ensure this, we will feature in these training modules 
examples of recent and ongoing NIH-funded microbiology and immunology research 
programs devoted to drug and vaccine development for infectious diseases at 
Colorado Statue University. 
There will be two instructional sequences of modules, ``Improving the Reproducibility of Experimental Data Recording'', with eleven training modules, 
and ``Improving the Reproducibility of Experimental Data Pre-Processing", with nine training modules.
The R programming language, and an ecosystem of
related reproducibility tools, will form the technical basis for implementation
training modules in these sequences, while modules on principles and examples will be accessible to biomedical researchers regardless of programming knowledge.  
These training modules will be collectively
published as an open online book using the \textit{bookdown} technology, leveraging literate programming technology. Each
module will form a chapter of this book, and will feature an embedded YouTube
video of 10--25 minutes, with accompanying text in the book to provide trainees
with a more detailed written reference they can refer to after completing the
video module.  Each module's chapter will conclude with practical exercises or
open discussion questions to complement the material taught in the video. 
To
ensure this material is completely free and open to researchers in the United
States, we will publish this online book, and its embedded videos and additional content, under a Creative Commons
license.

\end{document}
