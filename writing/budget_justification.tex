\documentclass[pdftex,english,11pt,parskip=half]{scrartcl}
\usepackage{palatino}
\usepackage{mathpazo}
\usepackage[margin=0.7in]{geometry}
%\usepackage{parskip}
\usepackage[compact]{titlesec}
\usepackage{amsmath,amssymb}
\usepackage{graphicx}
\usepackage{babel}
\usepackage{framed}
\usepackage{wrapfig}
\usepackage{subfig}
\usepackage[labelfont=bf,font=small,format=plain]{caption}
\usepackage{doi}
\usepackage{booktabs}
\usepackage{longtable}
\usepackage{multirow}
\usepackage[table]{xcolor}
\usepackage{wrapfig}
\usepackage{colortbl}
\usepackage{pdflscape}
\usepackage{tabu}
\usepackage{threeparttable}
\usepackage{threeparttablex}
\usepackage{array}
\usepackage[normalem]{ulem}
\usepackage{makecell}
\usepackage{float}
%\usepackage[authoryear]{natbib}
\usepackage[numbers]{natbib}
\usepackage{url,hyperref,color}
\definecolor{darkblue}{rgb}{0.0,0.0,0.75}
\hypersetup{colorlinks,breaklinks,
            linkcolor=darkblue,urlcolor=darkblue,
            anchorcolor=darkblue,citecolor=darkblue}
\newcommand{\fixme}[1]{{\color{red} #1}}
\renewcommand\thesection{\Alph{section}}
\renewcommand{\familydefault}{\sfdefault}
\newcommand{\tabitem}{~~\llap{\textbullet}~~}
\begin{document}
\addtokomafont{section}{\large}
\def\bf{\normalfont\bfseries}
\pagestyle{empty}

\textbf{Budget justification} \ \\
% Total direct costs over the three years of the project: $250,000
% May want to distribute as: $112,500 Year 01, $112,500 Year 02, $25,000 Year 03
% Modules should be published by the end of the second year. The third year 
% can be used for futher evaluation and refining of these modules.

{\large \textbf{Personnel}} \\

\noindent \textbf{Brooke Anderson,} \textit{Ph.D., Principal Investigator, x academic person-months, x summer person-months (25\% effort) in Years 01 through 02, x academic person-months, x summer person-months (15\% effort) in Year 03.} Dr. Anderson is an Assistant Professor of Epidemiology in the Department of Environmental \& Radiological Health Sciences at Colorado State University, with an affiliate position at the Department of Statistics. She is an expert in R programming and has created and published several open-source R packages, in particular to facilitate environmental epidemiological research. She has experience creating R programs to work with large data, including climate model output and large weather datasets, as well as programs that interface with open web-based datasets. She is the co-instructor of a series of Massive Open Online Courses on \textit{Mastering Software Development in R} through Coursera and an associated open online book. Dr. Anderson will lead the development and refinement of all training modules developed through this grant, including through supervising the development and integration of training materials from co-investigators. She will also lead user testing and other evaluation of all developed modules to ensure the developed modules are clear, effective, and well-matched to meet the needs of biological researchers from a variety of scientific backgrounds, including those new to programming. She will coordinate the contributions of all other Co-Is in helping to evaluate and disseminate the training materials. She will travel to conferences in Years 02 and 03 of the project to help disseminate the materials to our key audience. She will supervise the student hourly, who will assist her in Years 01 and 02 in the technical development and online publishing of the online book that will contain all training materials. 

\noindent \textbf{Michael Lyons,} \textit{Ph.D., Co-Investigator, x academic person-months, x summer person-months (5\% effort) in Years 01--03, x academic person-months.} Michael Lyons is an Assistant Professor at Colorado State University (CSU), where he works on the computational biology and pharmacology of tuberculosis (TB) infection and treatment in experimental animal models and TB patients. Prior to joining CSU full-time in 2011, he was a software engineer in the computer industry for 12 years, and prior to that, a theoretical physicist. Through a K25 award, he obtained significant classroom and hands-on training and exposure to laboratory methods related to drug and vaccine development for TB, providing him with a solid understanding of how preclinical and clinical data are used for evidence-based decision making in the biomedical sciences. He is highly attuned to the problems that this project aims to address, and he has a clear understanding of the practical limitations and challenges for both the laboratory scientist and data analyst. In this project, he will be one of the co-authors of the online book containing the training modules. He will contribute to the development, testing, and refinement of materials for all training modules, with a particularly strong role in helping to create the ``Implementation" and ``Principles" modules. He will help plan and attend all of the biannual day-long pilot testing sessions at CSU. He will help disseminate the final training materials among our target audience. 

\noindent \textbf{Marcela Henao-Tamayo,} \textit{Ph.D., Co-Investigator, x academic person-months, x summer person-months (5\% effort) in Years 01--03.} Marcela Henao-Tamayo is an Assistant Professor at Department of Microbiology, Immunology \& Pathology, College of Veterinary Medicine and Biomedical Sciences, and the Co-Director of CSU-Flow Cytometry Facility at Colorado State University. She studies the immunopathogenesis of tuberculosis using animal models to evaluate the role of different types of T cells and Myeloid Derived cells in tuberculosis and BCG vaccination (Bacille Calmette Guerin, the only approved vaccine against TB). She has tested numerous vaccine candidates evaluating the immune response they elicit in association with protection against tuberculosis disease. She is interested in how existing tools for computational reproducibility can be applied to data recording and pre-processing in her own research laboratory, and she and Dr. Anderson (PI) co-advise a graduate student who is integrating open-source R software into the regular practice of Dr. Henao-Tamayo's research work, including through implementation of reproducible automated gating of flow cytometry data. In this project, she will be one of the co-authors of the online book containing the training modules. She will contribute to the development, testing, and refinement of materials for all training modules, with a particularly strong role in helping to create the ``Principles" and ``Examples" modules, with some of the ``Example" modules focused on improving reproducibility in her own research projects, particularly in the sequence on data pre-processing. She will serve as one of the first testers of the ``Implementation" modules, to help us determine their clarity, relevance, and usefulness for a laboratory-based scientist. She will attend the biannual day-long pilot testing sessions at CSU, and members of her research group will attend these sessions as pilot testers. She will help disseminate the final training materials among our target audience, including through helping prepare the workshops and presentations to be given at conferences during the grant to help disseminate the materials. 

\noindent \textbf{Mercedes Gonzalez-Juarrero,} \textit{Ph.D., Co-Investigator, x academic person-months, x summer person-months (5\% effort) in Years 01--03.} Mercedes Gonzalez-Juarrero is an Associate Professor in the Department of Microbiology, Immunology \& Pathology at Colorado State University. Her research interest is in studying the basic nature of the cell mediated immune response to mycobacteria infections. During the last ten years, her research group has undertaken studies to investigate the emergence of immunosuppression during pulmonary tuberculosis, with the primary goal of learning how and where to target the latently infected host to fully recover the antimicrobial activity of the infected cell, and how to use this information in the context of current chemotherapeutic and multidrug resistant TB infections. Dr. Gonzalez-Juarrero became particularly interested in how to improve the reproducibility, transparency, and efficiency of experimental data recording within her research projects when she attended Dr. Anderson's (PI) CSU course on \textit{R Programming for Research} in Fall 2017 and learned about the principles of structured data formats, including the ``tidy" data format now popular with statisticians. This realization led to important discussions with Drs. Anderson and Lyons regarding data recording and understanding the role of each scientist from data recording through to analysis. In this project, she will be one of the co-authors of the online book containing the training modules. She will contribute to the development, testing, and refinement of materials for all training modules, with a particularly strong role in helping to create the ``Principles" and ``Examples" modules, with some of the ``Example" modules focused on improving reproducibility in her own research projects, particularly in Sequence 1 (data recording). She will serve as one of the first testers of the ``Implementation" modules, to help us determine their clarity, relevance, and usefulness for a laboratory-based scientist. She will attend the biannual day-long pilot testing sessions at CSU, and members of her research group will attend these sessions as pilot testers. She will help disseminate the final training materials among our target audience, including through helping prepare the workshops and presentations to be given at conferences during the grant to help disseminate the materials.

\noindent \textbf{Gregory Robertson,} \textit{Ph.D., Co-Investigator, x academic person-months, x summer person-months (5\% effort) in Years 01--03.} Dr. Gregory Robertson is an Assistant Professor of Microbiology, Immunology and Pathology at Colorado State University. Dr. Robertson has more than 20 years of classical and clinical microbiology experience with emphasis in antibacterial discovery and mode-of-action studies for novel and existing classes of antimicrobials. This includes efforts in academia, and also with larger pharmaceutical corporations (Eli Lilly and Co) and smaller bio-pharmaceutical groups (Cumbre Pharmaceuticals). His current research is focused on \textit{Mycobacterium tuberculosis} host-pathogen interactions and the development and application of novel preclinical animal models to further anti-tuberculosis drug development and evaluate drug resistance. In the context of improving reproducibility in biomedical research, Dr. Robertson is particularly passionate about the perils of using spreadsheets with embedded macros as a tool for recording and analyzing experimental data. In this project, he will be one of the co-authors of the online book containing the training modules. He will contribute to the development, testing, and refinement of materials for all training modules, with a particularly strong role in helping to create the ``Principles" and ``Examples" modules, with some of the ``Example" modules focused on improving reproducibility in her own research projects. He will serve as one of the first testers of the ``Implementation" modules, to help us determine their clarity, relevance, and usefulness for a laboratory-based scientist. He will attend the biannual day-long pilot testing sessions at CSU, and members of his research group will attend these sessions as pilot testers. He will help disseminate the final training materials among our target audience, including through helping prepare the workshops and presentations to be given at conferences during the grant to help disseminate the materials. 

\noindent \textbf{Julie Maertens.} \textit{Senior evaluator at the CSU STEM Center, Key Personnel, x academic person-months, x summer person-months (0.85\% effort) in Years 01--03.} Julie Maertens will assist in the design and implementation of project evaluation throughout this project. Julie Maertens is a senior evaluator at the Colorado State University STEM Center, which assists and collaborates with faculty involved with STEM education-based research and programming in designing and carrying out evaluations. Julie will only be involved in the project to assist in planning and implementing evaluation, and her percent effort is capped at 0.85\% to reflect the RFA's budget restriction of \$3,000 total on program evaluation, including salary support.
% Maximum budget for evaluation costs: $3,000 over the project period

\noindent \textbf{Undergraduate student hourly.} \textit{20 hours / week, \$14 / hour, in Years 01--02.} The undergraduate student will assist Dr. Anderson in Years 01 and 02 of the grant, particularly with the technical aspects of creating and publishing online the book that will contain all training module materials. We plan to recruit a student who is willing to collaborate with Dr. Anderson and other members of the team through GitHub to create these materials. Specific roles will include working on formatting for the final book, helping to post video lectures to YouTube and adding code to embed them to the book, setting up online surveys based on the questions developed for evaluation, and helping to create implementations of the additional educational materials like online quizzes and applied exercises based on the content developed by the PIs and Co-Is. 

{\large \textbf{Travel}} \\

\noindent \textbf{Domestic travel.} Domestric travel funds are requested for four domestic travel expenses: (1) Travel for the PI and one co-I to attend either the UseR or RStudio Conference in Year 01, to ensure the training materials include the most cutting-edge implementation methods for improving computational reproducibility of research and to learn of the latest developments for using R's \textit{bookdown} interface to create and disseminate free and open training materials; (2) Travel for the PI and one co-I to attend the American Society for Microbiology Conference in June 2020 (Year 02 of project period) to lead a workshop based on the developed training materials and to present a poster or oral presentation on the project results to help disseminate these results to laboratory-based biomedical researchers; (3) travel for the PI to attend the American Association of Immunologists Meeting in Year 02 of the project, to present a paper or oral presentation to help disseminate the availability of the training materials among our target audience; and (4) Travel for PI to attend a Program Meeting in Year 03 of the project. 
\begin{itemize}
\item Year 01: \$3,400
\item Year 02: \$5,100
\item Year 03: \$3,000
\end{itemize}
% The conference travel budget estimates $400 per flight, $175 per night at a hotel for four nights, and $75 per day for food for five days, plus $80 for the airport shuttle, plus $50 for checked luggage, plus $100 for taxis then rounds up. 
% The SC meeting travel budget estimates $300 for flight, $175 per night at a hotel for 2.5 nights, and $75 per day for food for three days, then rounds down (presumably some meals would be covered).

\noindent \textbf{International travel.} Travel for the PI to attend the International Society for Advancement of Cytometry in Year 03 of the project to present a paper or oral presentation to help disseminate the availability of the training materials among our target audience. 
\begin{itemize}
\item Year 01: \$0
\item Year 02: \$0
\item Year 03: \$2,300
\end{itemize}

{\large \textbf{Materials and Supplies}} \\ Annual funds (\$300/year) are requested for screen-capture software (e.g., \textit{Camtasia}), other software, and books to facilitate the proposed training module development. \\

{\large \textbf{Publication costs}} \\ Funds (\$1000 in Year 03) are requested for publication fees for the paper we plan to submit to describe the modules and help disseminate their availability to our target audience. \\

\noindent \textbf{Conference Registrations.} Funds are budgeted for (1) Registration for PI and one co-I to attend either the UseR or RStudio Conferences in Year 01 of the project period, to ensure cutting-edge implementation methods for improving computational reproducibility of research are included in the training materials and to learn the latest techniques for using R's \textit{bookdown} interface to create and disseminate free and open training materials; (2) Registration for the PI and one co-I to for the American Society for Microbiology Conference in Year 02 of the project period, for which they will apply to lead a workshop based on the developed training materials and to present a poster or oral presentation on the project results to help disseminate these results to laboratory-based biomedical researchers; (3) Registration for the PI to attend the American Association of Immunologists Meeting in Year 02 of the project period to present a poster or oral presentation on the project results to help disseminate these results to laboratory-based biomedical researchers; and (4) Registration for PI to attend the International Society for Advancement of Cytometry in Year 03 of the project period to present a poster or oral presentation on the project results to help disseminate these results to laboratory-based biomedical researchers.
\begin{itemize}
\item Year 01: \$1,600
\item Year 02: \$2,400
\item Year 03: \$800
\end{itemize}

\noindent \textbf{Hospitality.} Funds are budgeted each year to provide breakfast, lunch, coffee, and snacks for biannual user testing days with up to 20 faculty, research associates, postdoctoral fellows, graduate students, and undergraduate students from Colorado State University (budgeted at \$425/day). Colorado State University provides room reservations free to faculty for similar events, and so funds to rent a space are not required (See Letter of Support, Dr. Jac Nickoloff). Our budget for this is based on previous costs of similar day-long user testing sessions lead by the PI at Colorado State University.
\begin{itemize}
\item Year 01: \$850
\item Year 02: \$850
\item Year 03: \$850
\end{itemize}
% About $13 per "box" of coffee, serves about 12
% About $9 per person for catering from Spoons (for lunch)
% About $6 per person for catering from Panera for breakfast
% About $4 per person for snacks seems reasonable
% Breakfast for 20 x 2; Lunch for 20 x 2; Snacks for 20 x 2; Coffee for 20 x 4; 
% Total hospitality budget for extended user testing (round up): $800 
% Snacks for 15 x 3; Coffee for 15 x 3
% Total hospitality budget for each shorter user testing (round up): $75 



\end{document}