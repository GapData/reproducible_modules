\documentclass[pdftex,english,11pt,parskip=half]{scrartcl}
\usepackage{palatino}
\usepackage{mathpazo}
\usepackage[margin=0.7in]{geometry}
%\usepackage{parskip}
\usepackage[compact]{titlesec}
\usepackage{amsmath,amssymb}
\usepackage{graphicx}
\usepackage{babel}
\usepackage{framed}
\usepackage{wrapfig}
\usepackage{subfig}
\usepackage[labelfont=bf,font=small,format=plain]{caption}
\usepackage{doi}
\usepackage{booktabs}
\usepackage{longtable}
\usepackage{multirow}
\usepackage[table]{xcolor}
\usepackage{wrapfig}
\usepackage{colortbl}
\usepackage{pdflscape}
\usepackage{tabu}
\usepackage{threeparttable}
\usepackage{threeparttablex}
\usepackage{array}
\usepackage[normalem]{ulem}
\usepackage{makecell}
\usepackage{float}
%\usepackage[authoryear]{natbib}
\usepackage[numbers]{natbib}
\usepackage{url,hyperref,color}
\definecolor{darkblue}{rgb}{0.0,0.0,0.75}
\hypersetup{colorlinks,breaklinks,
            linkcolor=darkblue,urlcolor=darkblue,
            anchorcolor=darkblue,citecolor=darkblue}
\newcommand{\fixme}[1]{{\color{red} #1}}
\renewcommand\thesection{\Alph{section}}
\renewcommand{\familydefault}{\sfdefault}
\newcommand{\tabitem}{~~\llap{\textbullet}~~}
\begin{document}
\addtokomafont{section}{\large}
\def\bf{\normalfont\bfseries}
\pagestyle{empty}

{\large \textbf{Human Subjects}}

\subsubsection*{Human Subjects Involvement and Characteristics}

For this project, we plan to use pilot testing to help us evaluate and improve the trianing modules. We ask for and record data on the number of and basic make-up (educational level, research roles) of the pilot testers and online users. We will also ask users to complete a survey to collect their feedback on the training modules and their effectiveness. 
For in-person educaiton pilot testingsessions  (CSU pilot testing sessions and ASM workshop), we will also send a similar follow-up survey by email approximately six months after the pilot testing session, to help evaluate the long-term outcomes of the training. For all in-person educational pilot testing sessions, the human subjects will be biomedical researchers (upper-level undergraduate students, graduate students, postdoctoral researchers, research associates, and faculty). We anticipate that the demographic make-up of these participants, both for the participants in the in-person educational pilot testing and for early users of the online material, will reflect the average demographics of our target audience of laboratory-based biomedical researchers, will typically be healthy, and will not include children, prisoners, or institutionalized individuals. We will conduct six educational pilot testing sessions at Colorado State University and one at a national microbiology conference, for each of which we expect to have approximately 20 participants. We anticipate there will be 50--75 early online users of the in-development online training material, who will be recruited through the planned dissemination efforts with social media and among our network of R users and biomedical researchers. The key benefits to participants are new training and career development. The risks to these participants will be minimal---no greater than the typical risks encountered in daily life.

\textbf{This research is \underline{exempt} human subjects research based on the National Institutes of Health's \underline{Exemption 1 (E1)} for ``research conducted in established or commonly accepted educational settings, involving normal educational practices, such as (i) research on regular and special education instructional strategies, or (ii) research on the effectiveness of or the comparison among instructional techniques, curricula, or classroom management methods" (Section 46.101 of 45 CFR 46).} The only human subjects in this project will be trainees in the educational settings of sessions where educational material will be presented in a typical, workshop-style educational setting (on-campus pilot testing and conference workshop) and trainees who use the online training modules we plan to develop. All interactions will therefore be in a typical educational setting, with normal educational practices, and with the purpose of gaining feedback on the frequency of use and effectiveness of the training materials developed through the grant. 

\end{document}