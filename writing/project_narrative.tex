\documentclass[pdftex,english,11.5pt,parskip=half]{scrartcl}
\usepackage{palatino}
\usepackage{mathpazo}
\usepackage[margin=0.55in]{geometry}
%\usepackage{parskip}
\usepackage[compact]{titlesec}
\usepackage{amsmath,amssymb}
\usepackage{graphicx}
\usepackage{babel}
\usepackage{framed}
\usepackage{wrapfig}
\usepackage{subfig}
\usepackage[labelfont=bf,font=small,format=plain]{caption}
\usepackage{doi}
\usepackage{booktabs}
\usepackage{longtable}
\usepackage{multirow}
\usepackage[table]{xcolor}
\usepackage{wrapfig}
\usepackage{colortbl}
\usepackage{pdflscape}
\usepackage{tabu}
\usepackage{threeparttable}
\usepackage{threeparttablex}
\usepackage{array}
\usepackage[normalem]{ulem}
\usepackage{makecell}
\usepackage{float}
%\usepackage[authoryear]{natbib}
\usepackage[numbers]{natbib}
\usepackage{url,hyperref,color}
\definecolor{darkblue}{rgb}{0.0,0.0,0.75}
\hypersetup{colorlinks,breaklinks,
            linkcolor=darkblue,urlcolor=darkblue,
            anchorcolor=darkblue,citecolor=darkblue}
\newcommand{\fixme}[1]{{\color{red} #1}}
\renewcommand\thesection{\Alph{section}}
\renewcommand{\familydefault}{\sfdefault}
\newcommand{\tabitem}{~~\llap{\textbullet}~~}
\begin{document}
\addtokomafont{section}{\large}
\def\bf{\normalfont\bfseries}
\pagestyle{empty}

{\large \textbf{Project Narrative:}}

% Three sentences

The reproducibility of biomedical research results is central to the aims of 
integrative biology and human health. This proposal identifies critical
points early in biomedical research workflows as targets to enhance data reproducibility,
and provides training modules that introduce  principles, guidelines, and tools
for a more rigorous approach to data recording and data pre-processing. 
Covering recent advances in the principles of improving reproducibility at these stages of the research workflow, as well as in tools from the R programming language ecosystem to implement these principles, these training modules will be made 
available as an innovative open online electronic book featuring a full 
collection of short video lectures, with supplemental text and practical exercises,
and are aimed at a wide range of laboratory-based scientists whose attention is 
rather to their experimental technique and collection of accurate data, and who 
may have little or no background in the use of general purpose software tools.

\end{document}
